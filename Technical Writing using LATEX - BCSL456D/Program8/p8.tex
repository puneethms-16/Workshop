% 8. Develop a LaTeX script to demonstrate the presentation of Numbered theorems, definitions,
% corollaries, and lemmas in the document.
\documentclass[a4paper,12pt]{article}
\usepackage{amsmath,amssymb,amsthm}
\newtheorem{theorm}{Theorm}[section]
\newtheorem{lemma}[theorm]{Lemma}
\newtheorem{definition}[theorm]{Definition}
\newtheorem{corollary}[theorm]{Corollary}
\begin{document}
\title{demonstring the number Theorm lemma definition corollary}
\author{Puneeth MS}
\date{\today}
\maketitle
\section*{Introdaction}
in this we are create the mathematicale theorm definition lemma corollary
\section{Theorms,lemma,corollary,and definition}
\subsection*{Theorm Example}
\begin{theorm}
if \(a\) and \(b\) are the reale number then thire sum is
\[a + b = b + a. \]  
\end{theorm}
\subsection*{lemma Example}
\begin{lemma}
    let \(a\) and \(b\) are real number,if \( a + b = 0\),then \( b = -a \). 
\end{lemma}
\subsection*{corollary Example}
\begin{corollary}
    if \( a + b = 0 \) and \(a=2\) then \(b = -2\)
\end{corollary}
\subsection*{definition Example}
\begin{definition}
    a number is called even if it is divaseble by 2
    \[
    n = 2k.
    \]
\end{definition}
\section{conclusion}
This document demonstring how the number Theorm are placed and make them efficent
\end{document}